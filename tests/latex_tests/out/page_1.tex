\documentclass[12pt]{article}
\usepackage{amsmath,amssymb,amsfonts}
\usepackage[utf8]{inputenc}
\usepackage{geometry}
\usepackage{xcolor}
\geometry{a4paper, margin=1in}

\begin{document}

\section*{\(\star\) MATH 151 Section 3.4 The Chain Rule}

Find derivatives of the following compositions of functions

(1) $h(x) = (2x-1)^2 = (2x-1)(2x-1) = 4x^2-4x+1$ \quad (Power Rule $\downarrow$) $((2x-1)^2)' = 2(2x-1)'$
So $h'(x) = 8x-4 = 2(4x-2) = 2(2x-1) \cdot 2$

(2) $h(x) = (3x+1)^2 = (3x+1)(3x+1) = 9x^2+6x+1$ \quad ($\downarrow$) $((3x+1)^2)' = 2(3x+1)$
So $h'(x) = 18x+6 = 2(9x+3) = 2(3x+1) \cdot 3$

(3) $h(x) = (x^2+1)^2 = (x^2+1)(x^2+1) = x^4+2x^2+1$ \quad ($\downarrow$) $((x^2+1)^2)' = 2(x^2+1)$
So $h'(x) = 4x^3+4x = 2(2x^3+2x) = 2(x^2+1) \cdot 2x$

(4) $h(x) = \sqrt{5x+1} = (5x+1)^{\frac{1}{2}}$
\begin{align*}
h'(x) &= \frac{1}{2}(5x+1)^{-\frac{1}{2}} \cdot 5 & h(x) &= \frac{3}{4x-1} = 3(4x-1)^{-1} \\
& & h'(x) &= -3(4x-1)^{-2} \cdot 4
\end{align*}

Note that each function $h$ is indeed compositions of functions, $h = f \circ g = f(g(x))$.
For the derivative of $f(g(x))$, you need $f'(g(x))$ as well as an extra term, namely $g'(x)$.

\section*{The Chain Rule}
Suppose the derivatives $g'(x)$ and $f'(g(x))$ both exist. Then the composition $F = f \circ g = f(g(x))$ has the derivative
\begin{align*}
F'(x) &= f'(g(x)) g'(x)
\end{align*}

% [Diagram: Composition of functions x -> g(x) -> f(g(x)) with f o g below]
\begin{center}
\fbox{
\parbox{0.7\textwidth}{
\centering
Diagram placeholder for:
$x \xrightarrow{g} g(x) \xrightarrow{f} f(g(x))$
with $f \circ g$ below the arrows.
}
}
\end{center}

In Leibniz notation, if $y = f(u)$ and $u = u(x)$ are both differentiable functions, then
\begin{align*}
\frac{dy}{dx} &= \frac{df}{du} \frac{du}{dx}
\end{align*}

% [Diagram: Composition of functions x -> u=u(x) -> f(u)=y]
\begin{center}
\fbox{
\parbox{0.7\textwidth}{
\centering
Diagram placeholder for:
$x \xrightarrow{u} u=u(x) \xrightarrow{f} f(u)=y$
with an arrow directly from $x$ to $f(u)=y$.
}
}
\end{center}

Is it really canceling?
$\frac{df}{\cancel{du}} \frac{\cancel{du}}{dx} = \frac{df}{dx} = \frac{dy}{dx}$
NO and YES in sect 3.10
for now

\end{document}