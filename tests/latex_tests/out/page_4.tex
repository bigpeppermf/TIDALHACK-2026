\documentclass[12pt]{article}
\usepackage{amsmath,amssymb,amsfonts}
\usepackage[utf8]{inputenc}
\usepackage{geometry}
\usepackage{xcolor} % Required for \textcolor
\geometry{a4paper, margin=1in}

\begin{document}

\section*{Example 2 Find $h'(4)$ for $h(x) = \sqrt{4+5f(x)}$ where $f(4)=9$ and $f'(4)=7$}
\begin{align*}
h(x) &= (4+5f(x))^{\frac{1}{2}} \\
h'(x) &= \frac{1}{2} (4+5f(x))^{-\frac{1}{2}} \cdot 5f'(x) \\
h'(4) &= \frac{1}{2} (4+5f(4))^{-\frac{1}{2}} \cdot 5f'(4) \\
&= \frac{1}{2} (4+5 \cdot 9)^{-\frac{1}{2}} \cdot 5 \cdot 7 \\
&= \frac{1}{2} (4+45)^{-\frac{1}{2}} \cdot 35 \\
&= \frac{1}{2} (49)^{-\frac{1}{2}} \cdot 35 \\
&= \frac{1}{2} \cdot \frac{1}{\sqrt{49}} \cdot 35 \\
&= \frac{1}{2} \cdot \frac{1}{7} \cdot 35 \\
&= \frac{35}{14} = \frac{5}{2}
\end{align*}

\section*{Derivative of $y=a^x$}
For any $a>0$, $e^{\ln a}=a$ (why?)
$a \xrightarrow{\ln} \ln a \xrightarrow{\exp} e^{\ln a} = a$
$\ln x$ is the inverse of $e^x$
By the chain rule,
\begin{align*}
\frac{d}{dx} a^x &= \frac{d}{dx} (e^{\ln a})^x \\
&= \frac{d}{dx} e^{(\ln a)x} \\
&= e^{(\ln a)x} \cdot \ln a \\
&= a^x \cdot \ln a
\end{align*}

\section*{Theorem The derivative of exponential function $y=a^x$ is}
\begin{align*}
(a^x)' &= \ln a \cdot a^x \\
(e^x)' &= \ln e \cdot e^x = e^x
\end{align*}
For example, the tangent line of $y=2^x$ at $x=0$ has the slope
$(2^x)' = \ln 2 \cdot 2^x \Rightarrow \ln 2 \cdot 2^0 = \ln 2$
% [Diagram: Graph of y=2^x with a tangent line at x=0, showing slope = ln 2]
\fbox{\parbox[c][3cm][c]{5cm}{\centering Graph Placeholder}}

\section*{Example 3}
(1) $f(x) = 4^{\cos \pi x}$
$X \mapsto \cos \pi x \mapsto 4^{\cos \pi x}$
\begin{align*}
f'(x) &= \ln 4 \cdot 4^{\cos \pi x} (\cos \pi x)' \\
&= \ln 4 \cdot 4^{\cos \pi x} (-\sin \pi x) \pi
\end{align*}
(2) $f(x) = x^6 4^{(x^6+2)}$
\begin{align*}
f' &= (6x^5) 4^{(x^6+2)} + x^6 (\ln 4 \cdot 4^{(x^6+2)} (6x^5))
\end{align*}
What about the derivative of $(x^x)$? $(x^x)' = \ln x \cdot x^x$? No

\section*{Logarithmic Differentiation}
\vspace{1cm} % Added to match original layout visually

\begin{center}
4
\end{center}

\end{document}