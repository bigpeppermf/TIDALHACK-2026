\documentclass[12pt]{article}
\usepackage{amsmath,amssymb,amsfonts}
\usepackage[utf8]{inputenc}
\usepackage{geometry}
\geometry{a4paper, margin=1in}
\begin{document}
\section*{MATH 151 Section 3.4 The Chain Rule}

Find derivatives of the following compositions of functions

\begin{align*}
\text{(1)} \quad h(x) &= (2x-1)^2 = (2x-1)(2x-1) = 4x^2 - 4x + 1 \quad &&\text{Power Rule} \\
\text{So } h'(x) &= 8x-4 = 2(4x-2) = 2(2x-1) \cdot 2 && ((2x-1)^2)' = 2(2x-1)
\\[1em]
\text{(2)} \quad h(x) &= (3x+1)^2 = (3x+1)(3x+1) = 9x^2+6x+1 \\
\text{So } h'(x) &= 18x+6 = 2(9x+3) = 2(3x+1) \cdot 3 && ((3x+1)^2)' = 2(3x+1)
\\[1em]
\text{(3)} \quad h(x) &= (x^2+1)^2 = (x^2+1)(x^2+1) = x^4+2x^2+1 \\
\text{So } h'(x) &= 4x^3+4x = 2(2x^3+2x) = 2(x^2+1) \cdot 2x && ((x^2+1)^2)' = 2(x^2+1)
\\[1em]
\text{(4)} \quad h(x) &= \sqrt{5x+1} = (5x+1)^{\frac{1}{2}} & h(x) &= \frac{3}{4x-1} = 3(4x-1)^{-1} \\
h'(x) &= \frac{1}{2}(5x+1)^{-\frac{1}{2}} \cdot 5 & h'(x) &= -3(4x-1)^{-2} \cdot 4
\end{align*}

Note that each function $h$ is indeed compositions of functions, $h = f \circ g = f(g(x))$.
For the derivative of $f(g(x))$, you need $\underline{f'(g(x))}$ as well as an extra term, namely $g'(x)$.

\section*{The Chain Rule}
Suppose the derivatives $g'(x)$ and $f'(g(x))$ both exist. Then the composition $F = f \circ g = f(g(x))$ has the derivative
\begin{align*}
F'(x) &= f'(g(x)) g'(x)
\end{align*}
% [Diagram: x maps to g(x) via g, which maps to f(g(x)) via f. A curved arrow labeled f o g goes from x to f(g(x)).]

In Leibniz notation, if $y = f(u)$ and $u = u(x)$ are both differentiable functions, then
\begin{align*}
\frac{dy}{dx} &= \frac{df}{du} \frac{du}{dx}
\end{align*}
% [Diagram: x maps to u=u(x) via u, which maps to f(u)=y via f. A curved arrow goes from x to y.]

Is it really canceling?
\begin{align*}
\frac{df}{\cancel{du}} \frac{\cancel{du}}{dx} = \frac{df}{dx} = \frac{dy}{dx}
\end{align*}
NO and YES in sect 3.10 for now

\hfill 1

\section*{The Power Rule combined with the Chain Rule (Generalized Power Rule)}
If $n$ is a real number then
\begin{align*}
\frac{d}{dx} (g(x))^n &= n(g(x))^{n-1} g'(x) \quad \text{or} \quad \frac{d}{dx} (u)^n = n u^{n-1} \frac{du}{dx}
\end{align*}

\section*{Example 1 Differentiate the following functions.}
\begin{enumerate}
    \item $y = (1+2x)^{10} = (1+2x)(1+2x)\ldots(1+2x)$
    $Y' = 10(1+2x)^9 \cdot 2$
    \item $y = \frac{1}{(1+2x)^{2024}} = (1+2x)^{-2024}$
    $Y' = -2024(1+2x)^{-2025} \cdot 2$
    \item $F(x) = \sqrt[3]{x^2+2x-3} = (x^2+2x-3)^{1/3}$
    $F'(x) = \frac{1}{3}(x^2+2x-3)^{-2/3}(2x+2)$
    \item $f(x) = \frac{1}{\sqrt[4]{x^2+x+1}} = (x^2+x+1)^{-1/4}$
    $f'(x) = -\frac{1}{4}(x^2+x+1)^{-5/4}(2x+1)$
    \item $y = (2x+1)^5 (x^3-x+1)^4$
    $Y' = ((2x+1)^5)' \cdot (x^3-x+1)^4 + (2x+1)^5 \cdot ((x^3-x+1)^4)'$
    $= 5(2x+1)^4 \cdot 2 (x^3-x+1)^4 + (2x+1)^5 \cdot 4 (x^3-x+1)^3 (3x^2-1)$
    \item $g(x) = \sin 2x$
    $x \xrightarrow{} 2x \xrightarrow{\sin} \sin(2x)$
    $g'(x) = \cos(2x) \cdot 2$
    \item $f(x) = \sin \frac{1}{x}$
    $x \xrightarrow{x^{-1}} \frac{1}{x} \xrightarrow{\sin} \sin\frac{1}{x}$
    $f' = \cos\left(\frac{1}{x}\right) \cdot (-x^{-2})$
\end{enumerate}

\vfill
\begin{center}2\end{center}

(8) $g(x) = \sin x^2 + (\cos x)^2$
\begin{align*}
g'(x) &= \cos x^2 \cdot 2x + 2\cos x \cdot (-\sin x)
\end{align*}

(9) $g(x) = \tan(\cos x)$
\begin{align*}
g'(x) &= \sec^2(\cos x)(\cos x)' \\
&= \sec^2(\cos x)(-\sin x)
\end{align*}

(10) $g(x) = \tan x \cos x$
\begin{align*}
g'(x) &= (\tan x)'\cos x + \tan x (\cos x)' \\
&= \sec^2 x \cos x + \tan x (-\sin x)
\end{align*}

(11) $G(t) = e^{7t \sin 2t}$
% [Diagram: Chain rule visualization for $e^{7t \sin 2t}$]
\begin{align*}
G'(t) &= e^{7t \sin 2t} (7t \cdot \sin 2t)' \\
&= e^{7t \sin 2t} (7 \sin 2t + 7t \cdot \cos 2t \cdot 2)
\end{align*}

(12) $h(t) = \left(\frac{t^4+5}{t^2+5}\right)^3$
\begin{align*}
h'(t) &= 3 \left(\frac{t^4+5}{t^2+5}\right)^2 \left(\frac{t^4+5}{t^2+5}\right)' \\
&= 3 \left(\frac{t^4+5}{t^2+5}\right)^2 \left(\frac{4t^3(t^2+5) - (t^4+5)2t}{(t^2+5)^2}\right)
\end{align*}

(13) $y = \sin^2(\cos 3x) = (\sin(\cos 3x))^2$
% [Diagram: Chain rule breakdown for $y = \sin^2(\cos 3x)$]
\begin{align*}
Y' &= 2 \sin(\cos 3x) (\sin(\cos 3x))' \\
&= 2 \sin(\cos 3x) \cos(\cos 3x) (\cos 3x)' \\
&= 2 \sin(\cos 3x) \cos(\cos 3x) (-\sin 3x) (3x)' \\
&= 2 \sin(\cos 3x) \cos(\cos 3x) (-\sin 3x) \cdot 3
\end{align*}

\section*{Example 2 Find $h'(4)$ for $h(x) = \sqrt{4+5f(x)}$ where $f(4)=9$ and $f'(4)=7$}

\begin{align*}
h(x) &= (4+5f(x))^{1/2} \\
h'(x) &= \frac{1}{2} (4+5f(x))^{-1/2} \cdot 5f'(x) \\
h'(4) &= \frac{1}{2} (4+5f(4))^{-1/2} \cdot 5f'(4) \\
&= \frac{1}{2} (4+5(9))^{-1/2} \cdot 5 \cdot 7 \\
&= \frac{1}{2} (4+45)^{-1/2} \cdot 35 \\
&= \frac{1}{2} (49)^{-1/2} \cdot 35 \\
&= \frac{1}{2} \cdot \frac{1}{7} \cdot 35 \\
&= \frac{5}{2}
\end{align*}

\section*{Derivative of $y=a^x$}
For any $a > 0$, $(e^{\ln a}) = a$ (why?)
$\qquad a \xrightarrow{\ln} \ln a \xrightarrow{\exp} e^{\ln a} = a$
$\qquad \ln x$ is the inverse of $e^x$

By the chain rule,
\begin{align*}
\frac{d}{dx} a^x &= \frac{d}{dx} (e^{\ln a})^x \\
&= \frac{d}{dx} (e^{x \ln a}) \\
&= e^{x \ln a} \cdot \ln a \\
&= a^x \cdot \ln a
\end{align*}

\section*{Theorem The derivative of exponential function $y=a^x$ is}
\begin{align*}
(a^x)' &= \ln a \ a^x \\
(e^x)' &= \ln e \cdot e^x = e^x
\end{align*}
For example, the tangent line of $y=2^x$ at $x=0$ has the slope
\begin{align*}
(2^x)' &= \ln 2 \cdot 2^x \\
\Rightarrow \ln 2 \cdot 2^0 &= \ln 2
\end{align*}

% [Diagram: Graph of y=2^x and its tangent line at x=0]
\begin{center}
\fbox{\rule{0pt}{3cm}\rule{5cm}{0pt}}
\end{center}

\section*{Example 3}
(1) $f(x) = 4^{\cos \pi x}$
$\qquad x \mapsto \cos \pi x \mapsto 4^{\cos \pi x}$
\begin{align*}
f'(x) &= \ln 4 \cdot 4^{\cos \pi x} (\cos \pi x)' \\
&= \ln 4 \cdot 4^{\cos \pi x} (-\sin \pi x) \pi
\end{align*}
(2) $f(x) = x^6 4^{(x^6+2)}$
\begin{align*}
f' &= 6x^5 4^{(x^6+2)} + x^6 \ln 4 \cdot 4^{(x^6+2)} (6x^5)
\end{align*}

What about the derivative of $(x^x)$? $(x^x)' = \ln x \cdot x^x$? No
Logarithmic Differentiation

\vspace*{\fill}
\begin{center}
4
\end{center}

Example 4 A table of values for $f, g, f'$, and $g'$ is given.
\begin{center}
\begin{tabular}{|c|c|c|c|c|}
\hline
$x$ & $f(x)$ & $g(x)$ & $f'(x)$ & $g'(x)$ \\
\hline
1 & 3 & 2 & 9 & 4 \\
\hline
2 & 1 & 8 & 8 & -2 \\
\hline
3 & 7 & 2 & 3 & -7 \\
\hline
\end{tabular}
\end{center}

\begin{enumerate}
    \item If $h(x) = f(g(x))$, find $h'(1)$
    \begin{align*}
        h'(x) &= f'(g(x)) \cdot g'(x) \\
        h'(1) &= f'(g(1)) \cdot g'(1) \\
        &= f'(2) \cdot 4 \\
        &= 8 \cdot 4 = 32.
    \end{align*}

    \item If $H(x) = g(f(x))$, find $H'(2)$.
    \begin{align*}
        H'(x) &= g'(f(x)) f'(x) \\
        H'(2) &= g'(f(2)) f'(2) \\
        &= g'(1) \cdot 8 \\
        &= 4 \cdot 8 = 32.
    \end{align*}
\end{enumerate}

Consider the cycle $(\cos x)' \to (\cos x)'' \to (\cos x)''' \to (\cos x)''''$
\begin{align*}
    -\sin x \quad -\cos x \quad \sin x \quad \cos x
\end{align*}

\quad \quad \quad \quad \quad \quad \quad \quad \quad \quad \quad \quad \quad \quad \quad \quad \quad \quad \quad \quad \quad \quad \quad \quad \quad \quad \quad \quad \quad \quad \quad \quad \quad $\downarrow$

\section*{Quiz 4}
Example 5 Find the $2023^{\text{th}}$ derivative of $y = \cos(2x)$.
\begin{align*}
    2020 + 3 &= 2023 \\
    2^{2023} &\sin(2x)
\end{align*}
The general derivative formula for a product is:
\begin{align*}
    (e^{-x}f(x))' &= e^{-x}(-f(x)+f'(x))
\end{align*}

Example 6 Find the $2024^{\text{th}}$ derivative of $f(x) = xe^{-x}$.
\begin{align*}
    f'(x) &= e^{-x}(-x+1) \\
    f''(x) &= e^{-x}(x-1-1) = e^{-x}(x-2) \\
    f'''(x) &= e^{-x}(-x+2+1) = e^{-x}(-x+3) \\
    f^{(4)}(x) &= e^{-x}(x-3-1) = e^{-x}(x-4) \\
    f^{(2023)}(x) &= e^{-x}(-x+2023) \\
    f^{(2024)}(x) &= e^{-x}(x-2024)
\end{align*}

Example 7 Find $\frac{d^{2025}}{dx^{2025}}\left(\frac{1}{x}\right)$.
\begin{align*}
    f(x) &= x^{-1} \\
    f'(x) &= -x^{-2} \\
    f''(x) &= 2 \cdot x^{-3} \\
    f'''(x) &= -3 \cdot 2 \cdot 1 x^{-4} \\
    f^{(4)}(x) &= 4 \cdot 3 \cdot 2 \cdot 1 x^{-5} \\
    f^{(2025)}(x) &= -2025! x^{-2026} \\
    1 \cdot 2 \cdot 3 \cdots 2025
\end{align*}
\end{document}